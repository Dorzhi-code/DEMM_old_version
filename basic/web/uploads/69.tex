\documentclass{article}
\usepackage[utf8]{inputenc}
\usepackage[russian]{babel}
\usepackage{setspace,amsmath}
\usepackage{amsmath}
\pagestyle{empty} 
\addto\captionsrussian{\def\refname{Литература:}}
\addcontentsline{toc}{section}{Литература:}
\usepackage[usenames]{color}
\usepackage{colortbl}
\begin{document}
	
	\begin{flushleft}
	{\bf 	УДК 517.95; 532.5 }   
	\end{flushleft}
	\begin{center}
	\textbf{Программный комплекс для нахождения многомерных интегралов методом Монте-Карло}
	\end{center}

    \textit{Ханхасаев Владислав Николаевич,} к.ф.-м.н, доцент кафедры высшей математики Восточно-Сибирского государственного университета технологий и управления, 670000, ул. Смолина д. 26, e-mail: \textcolor{blue}{ hanhvladnick@mail.ru.} 
    
    \textit{Жамцаев Никита Сергеевич, }аспирант кафедры прикладной математики и дифференциальных уравнений Бурятского государственного университета, моб.тел.: +79834335469, e-mail: \textcolor{blue}{ nikita91797@gmail.com.}  
    
    \vspace*{5mm}
    Методы Монте-Карло применяются для численного решения математических задач путем моделирования случайных величин. Наиболее важным свойством этих методов является сокращение расчета математических ожиданий задачи. Поскольку математические ожидания чаще всего являются обычными интегралами, центральное положение в теории методов Монте-Карло занимают методы вычисления интегралов \cite{litlink1}. 
    Преимущества стохастических методов особенно очевидны при решении задач больших размерностей, когда использование традиционных аналитических методов затруднительно или полностью невозможно. До появления ЭВМ методы Монте-Карло не могли стать универсальными численными методами, поскольку ручное моделирование случайных величин является очень трудоемким процессом.
    Суть метода состоит в том, что в задачу вводится случайная величина $\xi$, которая изменяется по правилу $p(\xi$). Случайная величина выбирается так, чтобы искомое в задаче значение $A$ становилось математическим ожиданием \cite{litlink2} величины $\xi$, то есть $M(\xi) = A$.
    Таким образом, искомое значение $A$ определяется только теоретически. Чтобы найти его численно, нужно использовать статистические методы. Для этого необходимо взять выборку случайных чисел $\xi_i$ объема $N$. Затем необходимо вычислить выборочное среднее случайной величины $\xi$ по формуле:
    
    \begin{center}
    $	\overline{\xi} = \frac{\sum_{i=1}^n \xi_i}{N}$
    \end{center}
    
	Вычисленное выборочное среднее значение берется как приближение к A: $\overline{\xi}\cong A$. Для получения результата с приемлемой точностью требуется множество статистических тестов. Теория методов Монте-Карло изучает способы выбора случайных величин $\xi$ для решения различных задач, а также способы уменьшения дисперсии случайных величин \cite{litlink3}.
	
	В ходе работы был cформулирован подробный алгоритм вычисления интегралов с использованием метода Монте-Карло. Для проведения численного эксперимента создан программный комплекс на языке C++, который с приемлемой точностью вычисляет интеграл любой кратности. Приводятся расчеты тестового примера.
	
	\textbf{Ключевые слова:} метод Монте Карло, кратные интегралы, численные методы, программные комплексы, случайные величины.
	




\begin{thebibliography}{}
	\bibitem{litlink1} Бусленко Н. П., Голенко Д. И., Соболь И. М., Срагович В. Г, Шрейдер., Ю. А. Метод статистических испытаний (метод Монте-Карло). — Государственное издательство физико-математической литературы, 1962 г. — 332 с.
	\bibitem{litlink2} Кибирев, В. В. Теория вероятностей и математическая статистика: учебно-методический комплекс для студентов специальности Прикладная математика и информатика / В. В. Кибирев ; Бурят. гос. ун¬т. - Улан-Удэ : Издательство БГУ, 2012. - 130 с.
	\bibitem{litlink3} Кибирев, В. В. Метод Монте-Карло. Цепи Маркова и случайные функции / В. В. Кибирев ; Бурят. гос. ун-т. - Улан-Удэ : Издательство БГУ, 2009. - 87 с.
\end{thebibliography}
	
	
	
\end{document}